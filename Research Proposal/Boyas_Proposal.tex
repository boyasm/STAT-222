\documentclass{article}
\usepackage[margin=1in]{geometry}
\usepackage{pdfpages}
\usepackage{url}
\usepackage[autostyle]{csquotes}
    
\begin{document}
\title{Research Proposal\\STAT 222: MA Capstone}
\author{Matt Boyas}

\date{February 3, 2014}
\maketitle
\section{Dataset}
I plan on using the Social Conflict in Africa Database (SCAD) as my primary data source.  The SCAD was prepared by Cullen Hendrix and Idean Salehyan for the program on Climate Change and African Political Stability (CCAPS) at the Robert S. Strauss Center for International Security and Law at the University of Texas at Austin.  The data and codebook are made available online for research and other purposes.\footnote{\url{https://www.strausscenter.org/scad.html}}  The SCAD project authors describe the dataset:

\blockquote{The Social Conflict in Africa Database (SCAD) includes protests, riots, strikes, inter-communal conflict, government violence against civilians, and other forms of social conflict not systematically tracked in other conflict datasets. SCAD currently includes information on over 7,900 social conflict events from 1990 to 2011.\footnote{Ibid.}}
Depending on the progression of the research, it could be advantageous to use secondary data sources to add more information into the SCAD.  Possible additional information and sources could include:
\begin{itemize}
\item information on a country's democracy/freedom level taken from either the Polity IV Project\footnote{Prepared by The Center for Systemic Peace, \url{http://www.systemicpeace.org/polity/polity4.htm}} or the Freedom House Freedom in the World Scores\footnote{Prepared by Freedom House, \url{http://www.freedomhouse.org/report/freedom-world-aggregate-and-subcategory-scores}}
\item information including data on major armed conflicts, territorial disputes, alliances, and world religions taken from from the Correlates of War Project\footnote{\url{http://www.correlatesofwar.org}}
\end{itemize}
Like the primary SCAD data, these supplementary datasets are made available online for research purposes.  Merging information from these -- or other -- sources into the SCAD will be undertaken as required by the project after consultation of Victoria and/or Christine.

\section{Research Question}
Using SCAD as a primary souce of data, I intend to address the following primary research question:

\begin{itemize}
\item[]\textit{What differentiates an episode of social conflict that results in deaths from an episode of social conflict that does not result in deaths?}
\end{itemize}
Decisions about secondary research avenues will be made in consultation with Victoria and/or Christine after significant progress has been made on the primary research question.  Secondary research questions, directly related to the conclusions of the primary research question, could include the following.
\begin{itemize}
\item\textit{Is there a way to predict the number of deaths that will result from an episode of social conflict?}

Note that this model could either forecast the number of deaths in absolute terms, or it could predict the number of deaths relative to the size of the conflict, which could be measured by the number of participants or the length of the conflict.
\item\textit{Are certain countries in Africa more susceptible to social conflict resulting in deaths?  If yes, what might account for the differences between countries?}

If analysis shows that certain countries are more susceptible to social conflict resulting in deaths, possible factors to investigate to explain the differences between countries could include, but are not limited to, the dominant country religion, central government type, and/or country freedom level.
\end{itemize}

\section{Figure/Table Titles \& Captions}
Possible figures include:
\begin{itemize}
\item Title: Deaths and No Deaths in African Social Conflict Events, 1990--2011

Caption/Description: This figure is a split bar-chart, with one bar per year, visually showing the number of conflicts resulting in deaths relative to the total number of conflicts per year. Each bar represents all conflicts in the specified year and is split into two pieces, one piece for conflicts resulting in deaths and one piece for conflicts resulting in no deaths.
\item Title: Distribution of the Number of Deaths in African Conflicts, 1990--2011

Caption/Description: This figure is a histogram showing the distribution of the number of deaths in all of the African social conflicts from 1990--2011.
\item Title: Distribution of the Number of Deaths Per Conflict Participant, 1990--2011

Caption/Description: This figure is a histogram showing the proportion of the number of deaths relative to the number of participants (i.e., deaths per participant) in the conflict in all of the African social conflicts from 1990--2011.

\item Title: Distribution of the Number of Deaths Relative to Conflict Length, 1990--2011

Caption/Description: This figure is a histogram showing the proportion of the number of deaths relative to the length of the conflict in days (i.e., deaths per day) in all of the African social conflicts from 1990--2011.
\end{itemize}
Possible tables include:
\begin{itemize}
\item Title: Top 10 Most Violent Conflicts, 1990--2011

Caption/Description: This table shows the top 10 most violent conflicts (ranked by number of deaths) along with important accompanying information such as country, dates, and the major conflict issue.

\item Title: Deaths and No Deaths by Political Regime/Dominant Religion/Freedom Level, 1990--2011

Caption/Description: This table shows the number of death-resulting conflicts and the number of zero-death resulting conflicts split by political regime type, dominant country religion, and/or country freedom level.\footnote{The specifics of this particular table depend on what additional data source(s), if any, I decide to merge into the SCAD.  If I end up adding multiple variables that could be appropriate for such a table, then I will include multiple tables in the final paper.}
\end{itemize}
Additional tables (and perhaps plots/figures) will appear depending on the success of the modeling effort and the type of model created.  Possibilities include a plot illustrating a linear regression, if done, or a table showing model coefficients and the results of associated significance hypothesis tests.


\end{document}