\documentclass{article}
\usepackage[margin=1in]{geometry}
\usepackage{pdfpages}
\usepackage{url}
\usepackage[autostyle]{csquotes}
    
\begin{document}
\title{Research Proposal\\STAT 222: MA Capstone}
\author{Matt Boyas}

\date{February 3, 2014}
\maketitle
\section{Dataset}
I plan on using the Social Conflict in Africa Database (SCAD) as my primary data source, prepared by Cullen Hendrix and Idean Salehyan for the program on Climate Change and African Political Stability (CCAPS) of the Robert S. Strauss Center for International Security and Law at the University of Texas at Austin.  The data and codebook are made freely available online for research and other purposes.\footnote{\url{https://www.strausscenter.org/scad.html}}  The SCAD project authors describe the dataset:

\blockquote{The Social Conflict in Africa Database (SCAD) includes protests, riots, strikes, inter-communal conflict, government violence against civilians, and other forms of social conflict not systematically tracked in other conflict datasets. SCAD currently includes information on over 7,900 social conflict events from 1990 to 2011.\footnote{Ibid.}}

Depending on where the project goes over the course of the semester, it could be advantageous to use secondary data sources to bring more information into the project.  Possible sources could include:
\begin{itemize}
\item information on a country's democracy/freedom level taken from either the Polity IV Project\footnote{Prepared by The Center for Systemic Peace, \url{http://www.systemicpeace.org/polity/polity4.htm}} or the Freedom House Freedom in the World Scores\footnote{Prepared by Freedom House, \url{http://www.freedomhouse.org/report/freedom-world-aggregate-and-subcategory-scores}}
\item information including data on major armed conflict, territorial disputes, alliances, and world religions taken from from the Correlates of War Project\footnote{\url{http://www.correlatesofwar.org}}
\end{itemize}
Like the primary SCAD data, these supplementary datasets are made available online for research purposes.  Merging information from these -- or other -- sources into the SCAD will be undertaken as the project requires with consultation of Victoria and/or Christine.

\section{Research Question}
Using SCAD as a primary souce of data, I would like to address the question:

\begin{itemize}
\item[]\textit{What differentiates an episode of social conflict that results in deaths from an episode of social conflict that does not result in deaths?}
\end{itemize}

Secondary research questions, directly related to the research and conclusions related to the first question, could include the following.  Decisions about secondary research avenues will be made in consultation with Victoria and/or Christine after significant progress has been made on the primary research question.
\begin{itemize}
\item\textit{Is there a way to predict the number of deaths that will result from an episode of social conflict?}

Note that this model could either forecast the number of deaths in absolute terms, or it could predict the number of deaths relative to the size of the conflict (which could be measured by the number of participants or the length of the conflict).
\item\textit{Are certain countries in Africa more susceptible to social conflict resulting in deaths?  Why might that occur?  Could it be related to the dominant country religion?  What about country government type/freedom level?}
\end{itemize}

\section{Figure/Table Titles \& Captions}
Possible figures include:
\begin{itemize}
\item Title: African Social Conflict Events and Those Resulting in Deaths, 1990--2011

Caption/Description: A split bar-chart, with one bar per year, visually showing the number of conflicts resulting in deaths relative to the total number of conflicts per year.
\item Title: Distribution of the Number of Deaths in African Conflicts, 1990--2011

Caption/Description: A histogram showing the distribution of the number of deaths in all of the African social conflicts from 1990--2011.
\item Title: Distribution of the Number of Deaths Per Conflict Participant, 1990--2011

Caption/Description: A histogram showing the proportion of the number of deaths relative to the number of participants (i.e. deaths per participant) in the conflict in all of the African social conflicts from 1990--2011.

\item Title: Distribution of the Number of Deaths Relative to Conflict Length, 1990--2011

Caption/Description: A histogram showing the proportion of the number of deaths relative to the length of the conflict in days (i.e. deaths per day) in all of the African social conflicts from 1990--2011.
\end{itemize}
Possible tables include:
\begin{itemize}
\item Title: Top 10 Most Violent Conflicts, 1990--2011

Caption/Description: A table showing the top 10 most violent conflicts (ranked by number of deaths) along with information such as country, dates, and the major conflict issue.

\item Title: Deaths/No-Deaths by Regime Type, 1990--2011

Caption/Description: A table showing the number of death-resulting conflicts and the number of zero-death resuting conflicts split by country political regime type or freedom level (depending on what data source, if any, I decide to merge into the conflict dataset).
\end{itemize}
Additional tables (and perhaps plots/figures) will appear depending on the success of the modeling effort and the type of model created.  Possibilities include a plot illustrating a linear regression, if done, or a table showing model coefficients and the results of associated significance hypothesis tests.


\end{document}