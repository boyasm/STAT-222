\documentclass{article}
\usepackage[margin=1in]{geometry}
\usepackage{pdfpages}
\usepackage{url}
\usepackage[autostyle]{csquotes}
    
\begin{document}
\title{Sunday Update\\STAT 222: MA Capstone}
\author{Matt Boyas}

\date{February 9, 2014}
\maketitle
This week I completed the following:
\begin{enumerate}
\item I completed exercises suggested in class relating to shell scripting and Emacs.  I also continued to become comfortable with Git and GitHub by commiting files and pushing them to my GitHub repository as I completed tasks relating to my term project.

\item I spent a large amount of time during the week trying to figure out how to get my XLS dataset file (which also has commas included in string variables) into a CSV or TXT file that can be used with our class software.  Normally I would just open Excel and hit \textit{File $\to$ Save As CSV} but that is not a very reproducabe action, and I wanted this file format change to be scripted.  I tried working with tools and packages for Python, the Shell, and Perl, but either I couldn't get the code to work or I did not have privileges to install the package on my remote workstation.  Eventually, I discovered the R package \textit{gdata} that enables you to read XLS files into R, relying on the Perl package that I was not able to directly install and use.  Then I wrote a simple R function to read in my primary dataset and save it as a tab-delimited text file that will be compatible with our class analysis tools. 

\item I worked through some of Wes McKinney's \textit{Python for Data Analysis} to get myself prepared for class this coming week.  I read my tab-delimited dataset (produced by the R code described above) into iPython Notebook.  Then I worked with the variable giving the number of deaths for each social conflict episode to create a dummy variable that is 0 if the conflict had no deaths and 1 if the conflict had at least one death (and NaN if the number of deaths were not known).

\item Finally, I spent a little time continuing my literature review of this research topic.
\end{enumerate}



\end{document}